%%%%%%%%%%%%%%%%%%%%%%%%%%%%%%%%%%%%%%%%%
% Lachaise Assignment
% LaTeX Template
% Version 1.0 (26/6/2018)
%
% This template originates from:
% http://www.LaTeXTemplates.com
%
% Authors:
% Marion Lachaise & François Févotte
% Vel (vel@LaTeXTemplates.com)
%
% License:
% CC BY-NC-SA 3.0 (http://creativecommons.org/licenses/by-nc-sa/3.0/)
% 
%%%%%%%%%%%%%%%%%%%%%%%%%%%%%%%%%%%%%%%%%

%----------------------------------------------------------------------------------------
%	PACKAGES AND OTHER DOCUMENT CONFIGURATIONS
%----------------------------------------------------------------------------------------

\documentclass{article}
\usepackage{amsthm}
\usepackage{listings}
\usepackage{xcolor}
\usepackage{graphicx}


\lstset{
	language=Python,
	basicstyle=\ttfamily\small,
	keywordstyle=\color{blue},
	stringstyle=\color{red},
	commentstyle=\color{purple},
	showstringspaces=false,
	numbers=left,
	numberstyle=\tiny\color{gray},
	breaklines=true,
	frame=single,
	captionpos=b
}
\newtheorem{definition}{Definition}[section]
\newtheorem{proposition}{Proposition}
\newtheorem{corolary}{Corolary}

%%%%%%%%%%%%%%%%%%%%%%%%%%%%%%%%%%%%%%%%%
% Lachaise Assignment
% Structure Specification File
% Version 1.0 (26/6/2018)
%
% This template originates from:
% http://www.LaTeXTemplates.com
%
% Authors:
% Marion Lachaise & François Févotte
% Vel (vel@LaTeXTemplates.com)
%
% License:
% CC BY-NC-SA 3.0 (http://creativecommons.org/licenses/by-nc-sa/3.0/)
% 
%%%%%%%%%%%%%%%%%%%%%%%%%%%%%%%%%%%%%%%%%

%----------------------------------------------------------------------------------------
%	PACKAGES AND OTHER DOCUMENT CONFIGURATIONS
%----------------------------------------------------------------------------------------

\usepackage{amsmath,amsfonts,stmaryrd,amssymb} % Math packages

\usepackage{enumerate} % Custom item numbers for enumerations

\usepackage[ruled]{algorithm2e} % Algorithms

\usepackage[framemethod=tikz]{mdframed} % Allows defining custom boxed/framed environments

\usepackage{listings} % File listings, with syntax highlighting
\lstset{
	basicstyle=\ttfamily, % Typeset listings in monospace font
}

%----------------------------------------------------------------------------------------
%	DOCUMENT MARGINS
%----------------------------------------------------------------------------------------

\usepackage{geometry} % Required for adjusting page dimensions and margins

\geometry{
	paper=a4paper, % Paper size, change to letterpaper for US letter size
	top=2.5cm, % Top margin
	bottom=3cm, % Bottom margin
	left=2.5cm, % Left margin
	right=2.5cm, % Right margin
	headheight=14pt, % Header height
	footskip=1.5cm, % Space from the bottom margin to the baseline of the footer
	headsep=1.2cm, % Space from the top margin to the baseline of the header
	%showframe, % Uncomment to show how the type block is set on the page
}

%----------------------------------------------------------------------------------------
%	FONTS
%----------------------------------------------------------------------------------------

\usepackage[utf8]{inputenc} % Required for inputting international characters
\usepackage[T1]{fontenc} % Output font encoding for international characters

\usepackage{XCharter} % Use the XCharter fonts

%----------------------------------------------------------------------------------------
%	COMMAND LINE ENVIRONMENT
%----------------------------------------------------------------------------------------

% Usage:
% \begin{commandline}
%	\begin{verbatim}
%		$ ls
%		
%		Applications	Desktop	...
%	\end{verbatim}
% \end{commandline}

\mdfdefinestyle{commandline}{
	leftmargin=10pt,
	rightmargin=10pt,
	innerleftmargin=15pt,
	middlelinecolor=black!50!white,
	middlelinewidth=2pt,
	frametitlerule=false,
	backgroundcolor=black!5!white,
	frametitle={Command Line},
	frametitlefont={\normalfont\sffamily\color{white}\hspace{-1em}},
	frametitlebackgroundcolor=black!50!white,
	nobreak,
}

% Define a custom environment for command-line snapshots
\newenvironment{commandline}{
	\medskip
	\begin{mdframed}[style=commandline]
}{
	\end{mdframed}
	\medskip
}

%----------------------------------------------------------------------------------------
%	FILE CONTENTS ENVIRONMENT
%----------------------------------------------------------------------------------------

% Usage:
% \begin{file}[optional filename, defaults to "File"]
%	File contents, for example, with a listings environment
% \end{file}

\mdfdefinestyle{file}{
	innertopmargin=1.6\baselineskip,
	innerbottommargin=0.8\baselineskip,
	topline=false, bottomline=false,
	leftline=false, rightline=false,
	leftmargin=2cm,
	rightmargin=2cm,
	singleextra={%
		\draw[fill=black!10!white](P)++(0,-1.2em)rectangle(P-|O);
		\node[anchor=north west]
		at(P-|O){\ttfamily\mdfilename};
		%
		\def\l{3em}
		\draw(O-|P)++(-\l,0)--++(\l,\l)--(P)--(P-|O)--(O)--cycle;
		\draw(O-|P)++(-\l,0)--++(0,\l)--++(\l,0);
	},
	nobreak,
}

% Define a custom environment for file contents
\newenvironment{file}[1][File]{ % Set the default filename to "File"
	\medskip
	\newcommand{\mdfilename}{#1}
	\begin{mdframed}[style=file]
}{
	\end{mdframed}
	\medskip
}

%----------------------------------------------------------------------------------------
%	NUMBERED QUESTIONS ENVIRONMENT
%----------------------------------------------------------------------------------------

% Usage:
% \begin{question}[optional title]
%	Question contents
% \end{question}

\mdfdefinestyle{question}{
	innertopmargin=1.2\baselineskip,
	innerbottommargin=0.8\baselineskip,
	roundcorner=5pt,
	nobreak,
	singleextra={%
		\draw(P-|O)node[xshift=1em,anchor=west,fill=white,draw,rounded corners=5pt]{%
		Question \theQuestion\questionTitle};
	},
}

\newcounter{Question} % Stores the current question number that gets iterated with each new question

% Define a custom environment for numbered questions
\newenvironment{question}[1][\unskip]{
	\bigskip
	\stepcounter{Question}
	\newcommand{\questionTitle}{~#1}
	\begin{mdframed}[style=question]
}{
	\end{mdframed}
	\medskip
}

%----------------------------------------------------------------------------------------
%	WARNING TEXT ENVIRONMENT
%----------------------------------------------------------------------------------------

% Usage:
% \begin{warn}[optional title, defaults to "Warning:"]
%	Contents
% \end{warn}

\mdfdefinestyle{warning}{
	topline=false, bottomline=false,
	leftline=false, rightline=false,
	nobreak,
	singleextra={%
		\draw(P-|O)++(-0.5em,0)node(tmp1){};
		\draw(P-|O)++(0.5em,0)node(tmp2){};
		\fill[black,rotate around={45:(P-|O)}](tmp1)rectangle(tmp2);
		\node at(P-|O){\color{white}\scriptsize\bf !};
		\draw[very thick](P-|O)++(0,-1em)--(O);%--(O-|P);
	}
}

% Define a custom environment for warning text
\newenvironment{warn}[1][Warning:]{ % Set the default warning to "Warning:"
	\medskip
	\begin{mdframed}[style=warning]
		\noindent{\textbf{#1}}
}{
	\end{mdframed}
}

%----------------------------------------------------------------------------------------
%	INFORMATION ENVIRONMENT
%----------------------------------------------------------------------------------------

% Usage:
% \begin{info}[optional title, defaults to "Info:"]
% 	contents
% 	\end{info}

\mdfdefinestyle{info}{%
	topline=false, bottomline=false,
	leftline=false, rightline=false,
	nobreak,
	singleextra={%
		\fill[black](P-|O)circle[radius=0.4em];
		\node at(P-|O){\color{white}\scriptsize\bf i};
		\draw[very thick](P-|O)++(0,-0.8em)--(O);%--(O-|P);
	}
}

% Define a custom environment for information
\newenvironment{info}[1][Info:]{ % Set the default title to "Info:"
	\medskip
	\begin{mdframed}[style=info]
		\noindent{\textbf{#1}}
}{
	\end{mdframed}
}
 % Include the file specifying the document structure and custom commands

%----------------------------------------------------------------------------------------
%	ASSIGNMENT INFORMATION
%----------------------------------------------------------------------------------------

\title{Introduction to Graph Theory} % Title of the assignment

\author{Giovani Tavares\\ \texttt{giovanitavares@usp.br}} % Author name and email address

\date{University of Sao Paulo --- 2025.1} % University, school and/or department name(s) and a date

%----------------------------------------------------------------------------------------

\begin{document}

\maketitle % Print the title

%----------------------------------------------------------------------------------------
%	INTRODUCTION
%----------------------------------------------------------------------------------------

\section{February 25th, 2025} % Unnumbered section


\begin{itemize}
	\item Bondy \& Murty
	\item Diestel
	\item Combinatoria (Botler)
	\item SageMath
\end{itemize}

\subsection{Basic Concepts}

\begin{definition}[Grid]
	\label{def:grade}
	The grid $G_{m,n}$  is the graph (V,E) such that:
	\begin{itemize}
		\item $V = [m] \times [n] = {(x,y) | x \in [m], y \in [n]}$
		\item $E = \{(u,v), (x,y):  (u = x, \text{  and  } |v-y| = 1) \text{  or  } (v = y \text{  and  } |u -x| = 1)\} $
	\end{itemize}
\end{definition}



\begin{definition}[Cycle]
	\label{def:cycle}
	
	The cycle $G_{m,n}$  is the graph (V,E) such that:
	\begin{itemize}
		\item $V = [n]$
		\item $E = \{(x, (x \mod n) + 1\}) :  x \in [n]\}$
		
	\end{itemize}
	
\end{definition}


\begin{definition}[Order]
\label{def:order}
	The order of a graph $G$  is its number of vertices.
	\begin{itemize}
		\item $v(G) = |V|$
	\end{itemize}
\end{definition}


\begin{definition}[Size]
	\label{def:size}
	The size of  a graph $G$  is its number of edges.
	\begin{itemize}
		\item $e(G) = |E|$
	\end{itemize}
\end{definition}


\subsection{Adjascency and Incidency}

If $e = uv$, we say that $e$ links $u$ and $v$. We also say:

\begin{itemize}
	\item $u$ and $v$ are adjascent
	\item $u$ and $v$ are neighbors
	\item $e$ is incicent to $u$ and $v$
	\item We also say that two egdes that share the same vertex are adjascent. 
	\item Two edges that share both vertices are parallel or multiple
	\item An edge that links a vertex to itself is called a lasso
	\item A pair of vertices or edges that are not adjascent are called independent
	\item A set of vertices such that every pair or two vertices are independent is called an independent set or stable
\end{itemize}


\subsection{Degrees of a Vertex}

The degree of a vertex, denoted by $d_G(v)$ is the number of edges (of $G$) incident to $v$.

\begin{itemize}
	\item The minimum degree of $G$ is $\delta(G) = min\{d(v) :  v \in V(G)\}$
	\item The maximum degree of $G$ is $\Delta(G) = max\{d(v) :  v \in V(G)\}$
	\item The average degree of $G$ is $\bar{d}(G) = \frac{1}{v(G)} \sum_{v \in V} d(v)$
\end{itemize}
%----------------------------------------------------------------------------------------
%	PROBLEM 1
%----------------------------------------------------------------------------------------


\section{February 27th, 2025}


\begin{itemize}
	\item  Entregar lista no overleaf
	\item Monitoria as quintas (13h)
\end{itemize}

- $E \subseteq \binom{V}{2}$


\begin{proposition}[Lema do Aperto de Mao]
	\label{def:lema_aperto_de_mao}
	For  every graph $G$, the sum of the degrees of its vertices is equal to twice its number of vertices:
	
	\begin{align}
		\sum_{v \in V(G)} d_{G}(v) = 2e(G) = 2|E(G)|
	\end{align}
	
	\textbf{Proof: Contagem Dupla}
	\begin{align}
	 	S &= \{ (e, u) \in E \times V :  u \in e  \}
	\end{align}
	
	As every edge is at precisely two elements of $S$, then $|S| = 2e(G)$. Furthermore, if $u \in V$, then $u$ is at $d(v)$ elements of $S$ . Then $S = $ 
	
\end{proposition}

\begin{corolary}
	Every graph has an even number of odd degreed vertices
	
	\textbf{Proof}
	
	Using the handshake lemma, we can do induction in $e(G)$
\end{corolary}


\begin{definition}
		If $d_G(u) = 0$, we say that $u$ is an isolated vertex
\end{definition}

\begin{proposition}
	If $G$ does not have any isolated vertex and $e(G) <  v(G)$ , then $G$ has at least two vertices with degree $1$.
	
	\textbf{Proof}
	We know $e(G) \leq v(G) - 1$. Using the handshake lemma, we have:
	
	\begin{align}
		2(v(G) - 1) \geq 2e(G) &=  \sum_{u \in V(G)} d(u) \\
		&= x + \sum_{u \in V(G), d(u) \geq 2} d(u) \\
		&\geq  x + \sum_{u \in V(G)} 2 \\
		&= x +  2(v(G) - x) \\
		&= 
	\end{align}
\end{proposition}


\begin{proposition}
	If  $e(G) \geq  v(G) +  1$ , then $G$ has at least one vertex with degree $\geq 3$.
	
	\textbf{Proof}
	Suppose, by contradiction, that $d(u) \leq 2$ for every $u \in V(G)$.
	
	\begin{align}
		e(G) &= \frac{1}{2} \sum_V d(u) \leq \frac{1}{2} \sum_V 2 = \frac{1}{2} 2V(G) = V(G) \\
		&\implies e(G) \leq V(G), \text{   which is a contradiction.} \\
	\end{align}
\end{proposition}


\section{Special Types of Graphs}

\begin{itemize}
	\item A graph is \textbf{simple} if is does not contain "arestas multiplas nem lacos"
	\item A graph is \textbf{complete} is a simple graph such that $G=(V,E) : E = \binom{V}{2}$
	\item A complete graph with $n$ vertices is denoted by $K_n$
	\item $G$ is \textbf{empty} if $V(G) = E(G) = \emptyset$
	\item $G$ is \textbf{trivial} if $E(G) = \emptyset$
	\item We say that $G$ is \textbf{$K-regular$} if $d(u) = k$ for all and every $u \in V(G)$
	\item $G$ is regular if it is $K-regular$ for some $k \in N$
	\item Every complete graph is regular
	\item $K_n$ is $(n-1)-Regular$
	\item $G$ is \textbf{bipartite} if $V(G)$ can be partitioned into two distinct sets $X$ and $Y$ such that every edge has a vertex in $X$ and another one in $Y$. Such partition is the bipartition of $G$. 
	\item The bipartite complete graph with a bipartition $(X,Y)$ is the graph $G$ such that $V(G) = X U Y \text{  and  } E(G) = X \times Y$. There is no way for a graph to be complete bipartite, just the other way around. The notation for a bipartite complete graph is $K_{n,m}$, with $|X| = n$ and $|Y| = m$.
	\item The complement of $G$, denoted by $\bar{G}$, is the graph such that $V(\bar{G}) = V(G)$ and $E(\bar{G}) = \{  xy \in \binom{V}{2} : xy \notin E(G)\}$. $\bar{G}$ is basically the graph that has all the edges that $G$ does not have. 
\end{itemize}


\begin{proposition}
	Every graph $G$ such that $v(G) \geq 2$ has two vertices with the same degree.
	
	\textbf{Proof by induction}
	
	Induction hypothesis: every graph with $n-1$ vertices has two vertices with the same degree.
	
	Suppose $v(G) \geq 3$, we have two cases:
	
	\begin{itemize}
		\item i. $G$ has a vertex $u$ such that $d(u) = 0$. Hence, by the induction hypothesis, $G'= G - u$ also has two vertices with the same degree.
		
		\item $d(u) \geq 1$ for all $u \in V$
	\end{itemize}
	
	
	
	
\end{proposition}


It is easy to computationally test whether a graph is bipartite. (Exercise)
\left( \left( 

%----------------------------------------------------------------------------------------

\end{document}
